\documentclass{article}
\usepackage[utf8]{inputenc}
\usepackage[T2A]{fontenc}
\usepackage[serbian]{babel}
\usepackage{graphicx}
\usepackage{tabularx}
\usepackage{array}
\usepackage[a4paper]{geometry}
\usepackage{blindtext}
\usepackage{tocbibind} % Ubacuje literaturu u sadrzaj

\title{Key Features Net i njene primene}
\author{Milo\v s Vujasinovi\'c}
\date{Novi Sad, jul 2020}

\graphicspath{{img}}

% tabularx columns
\newcolumntype{Y}{>{\centering\arraybackslash}X}

% Variables
\newcommand{\titlelogosize}{2.4cm}

% custom make title
\makeatletter         
\renewcommand\maketitle{
	\thispagestyle{empty}
	\begin{tabularx}{\textwidth}{m{\titlelogosize}Ym{\titlelogosize}}	\includegraphics[width=\linewidth]{./pmflogo} &
		\begin{tabular}{@{}c@{}}
			УНИВЕРЗИТЕТ У НОВОМ САДУ\\
			ПРИРОДНО-МАТЕМАТИЧКИ ФАКУЛТЕТ\\
			ДЕПАРТМАН ЗА МАТЕМАТИКУ И\\
			ИНФОРМАТИКУ
		\end{tabular} &
		\includegraphics[width=\linewidth]{./unslogo}
	\end{tabularx}
	
	\vfill
	
	\begin{center}
		\LARGE
		\textbf{\@title}
		
		\medskip
		
		Seminarski rad
	\end{center}
	
	\vfill
	
	\begin{flushright}
		\large
		\textbf{\@author}
	\end{flushright}
	
	\bigskip
	\bigskip
	
	\begin{center}
		\large
		\@date
	\end{center}
	\pagebreak
	
	\clearpage
	\pagenumbering{arabic} 
}
\makeatother

\begin{document}
	\maketitle
	
	\tableofcontents
	
	\pagebreak
	\section*{Uvod}
	\addcontentsline{toc}{section}{\protect{}Uvod}
	
	Autoenkoderi su ve\' c godinama zlatni standard u smanjenju dimenzionalnosti podataka. Na\v cin na koji rade se pokazao kao veoma efikasan u otklanjanju \v suma i dopunjavanju podataka koji su o\v ste\' ceni. Motivisan datim primerima, ovaj rad poku\v sava da prika\v ze novi na\v cin gledanja na smanjene dimenzionalnosti podataka ma\v sinskim u\v cenjem. Zatim, primenom iznesenih ideja i nekih od principa koji se nalaze u osnovi autoenkodera se uvodi model neuronske mre\v ze koji za cilj da iz podataka koji se prosle\dj uju modelu izdvoji najbitnije odlike za klasifikaciju. Kroz ovaj postupak se tako\dj e razmatraju novi na\v cini treniranja i evaluacije modela, a na kraju se rezultati datog modela porede sa rezultatima tradicionalnih autoenkodera.
	
	\pagebreak
	\section*{Zaklju\v cak}
	\addcontentsline{toc}{section}{\protect{}Zaklju\v cak}
	
	\pagebreak
	\bibliographystyle{unsrt}
	\bibliography{references}
\end{document}